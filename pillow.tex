
\documentclass[12pt,a4paper]{article}

\RequirePackage[l2tabu, orthodox]{nag}

\usepackage[T1]{fontenc}
\usepackage[utf8]{inputenc}

\newcommand \versionno{\jobname.tex}

\usepackage{amsmath,amssymb,amsfonts,amscd,latexsym,amsthm}

\usepackage[margin=20mm]{geometry}

\usepackage{showkeys}

\usepackage[colorlinks=true,linktoc=all,linkcolor=black,citecolor=red,urlcolor=blue]{hyperref}

\begin{document}

\typeout{}\typeout{\versionno}\typeout{} \begin{center} \fbox{\texttt{\versionno\ -- } {\small \today\ }} \end{center}

\vspace{2mm}

\begin{center}
 \textit{ \Huge Conformal blocks and the pillow geometry}
\end{center}

\tableofcontents

\section{Introduction}

The aim of this note is to provide a derivation of Zamolodchikov's formula for sphere 4-point blocks by recursion in the channel dimension. Zamolodchikov's original derivation \cite{zam87b} is based on the heavy asymptotic limit. Several steps in the derivation are hard to justify, starting with the appearance of a differential equation. Moreover, it seems overly complicated to use the heavy limit, when we are really interested in the limit where the channel dimension is large, while the other dimensions and the central charge are finite. 

An geometrical interpretation of Zamolodchikov's formula was proposed in \cite{msz15}, in terms of the pillow geometry. Here we would like to use this interpretation for deriving the formula. By deriving we mean proving although not with full mathematical rigor: in particular, we assume without proof that the conformal blocks exist and that the recursion converges. By deriving we also mean justifying as simply as possible where the formulas come from. Ideally this would involve uniquely characterizing the pillow geometry in terms of properties that imply that blocks are ``nice'' in the pillow coordinates. 

It is relatively easy to determine the poles and residues of the $s$-channel conformal block $\mathcal{F}_\Delta$ as a function of the channel dimension $\Delta$. To prove the recursion, it remains to determine the $\Delta\to\infty$ asymptotic behaviour. The basic property that we need is 
\begin{align}
 \log \mathcal{F}_\Delta = O(\Delta)
\end{align}
This is not obvious at all from the pedestrian $z$-expansion of the block, which is of the type 
\begin{align}
 \mathcal{F}_\Delta = z^{\Delta-\Delta_1-\Delta_2}\sum_{k=0}^\infty f_k z^k \quad \text{where} \quad f_k =   O(\Delta^k)
 \label{fk}
\end{align}
Taking the log must therefore lead to cancellations of terms in $O(\Delta^{j\geq 2})$. To order $z^k$ we need $k-1$ cancellations. This is already nontrivial in the case $k=2$. A direct pedestrian approach to all orders looks doomed. Maybe it would be feasible in the AGT representation of blocks, but we are looking for a more elementary approach. 

We want to prove that the block is of the type 
\begin{align}
 \mathcal{F}_\Delta = (16q)^\Delta \Omega(q)S_\Delta(q) \qquad \text{with} \qquad S_\Delta(q) =  1 + \sum_{k=1}^\infty c_k q^k 
 \label{fd}
\end{align}
The property of the series $S_\Delta$ that we need can be formulated in several ways:
\begin{itemize}
 \item $\lim_{\Delta\to \infty} c_k = 0$
 \item Bulk fields are negligible as $\Delta\to\infty$, i.e. $\lim_{\Delta\to\infty} S_\Delta(q)$ is $\Delta_i$-independent (and can therefore be computed at $\Delta_i=0$). 
\end{itemize}
The function $q(z)= \frac{1}{16} \exp \lim_{\Delta\to \infty} \frac{\log \mathcal{F}_\Delta}{\Delta}$ can be deduced from the block. From this function, can we in principle deduce the pillow geometry?


\section{Pillow geometry}

The pillow is 
\begin{align}
 \left\{x,y\in \overline{\mathbb{C}}^2\middle| y^2=x(x-1)(x-z) \right\}\bmod (y\to -y)
\end{align}
The pillow metric is $dud\bar u$ where $u\equiv u+2\pi \equiv u+2\pi\tau$ is defined by 
\begin{align}
 du = \frac{1}{\theta_3(q)^2} \frac{dx}{y} \quad \text{with} \quad q=e^{i\pi\tau}
 \label{dudx}
\end{align}
Here $dx^2$ is the flat metric on the Riemann sphere $\overline{\mathbb{C}}$. The pillow is a flat torus, with the curvature concentrated at the 4 singularities $x=0,1,z,\infty$. The normalization factor $\theta_3(q)^2$ is determined by the condition that the $s$-cycle of the torus has length $2\pi$. By $s$-cycle we mean a cycle that separates $0,z$ from $1,\infty$:
\begin{align}
  \oint du = 2\pi \quad \implies \quad \oint \frac{dx}{y} =  2\pi\theta_3(q)^2 = 4K(z)
\end{align}
The correspondence between coordinates acts as:
\begin{align}
\renewcommand{\arraystretch}{1.5}
 \begin{array}{|c||c|c|c|c||c|}
 \hline 
  x,y & z & 0 & 1 & \infty & y\to -y 
  \\
  \hline 
   u &  \pi & 0 & \pi(\tau+1) & \pi\tau & u \to -u
   \\
   \hline 
 \end{array}
\end{align}
Special cases of the relations between $q,\tau,z$:
\begin{align}
 \renewcommand{\arraystretch}{1.5}
 \begin{array}{|c||c|c|c|}
 \hline 
  z & 0 & 1 & \infty 
  \\
  \hline 
  q & 0 & 1 & i 
  \\
  \hline 
  \tau & i\infty & 0 & 1 
  \\
  \hline 
 \end{array}
\end{align}
The factor $\Omega(q)$ in \eqref{fd} is deduced from the conformal transformation to $x,y\to u$. This explains in particular why it is factorized as a function of $\Delta_i$. 

\section{Ward identities}

In the $z$-coordinate, the behaviour \eqref{fk} of the block's coefficients follows from their calculation from Ward identities:
\begin{align}
 f_k = \sum_{|L|=|L'|=k} G^{-1}_{L,L'}(\Delta) g_L(\Delta,\Delta_1,\Delta_2)g_{L'}(\Delta,\Delta_4,\Delta_3)
\end{align}
where the sum is over creation operators $L=\prod_{i=1}^m L_{-n_i}$ of level $k=\sum n_i$, with 
\begin{align}
 \deg_\Delta g_L = m\leq k
\end{align}
For example, at level $k=1$, we have 
\begin{align}
 g_{L_{-1}} = \Delta + \Delta_1-\Delta_2 \quad , \quad G^{-1}_{L_{-1},L_{-1}} = \frac{1}{2\Delta} 
\end{align}
At level $k=2$, we have 
\begin{align}
G^{-1} = \frac{1}{16(\Delta-\Delta_{(2,1)})(\Delta-\Delta_{(1,2)})} \begin{bmatrix} 2+\frac{c}{4\Delta} & -3 \\ -3 & 4\Delta+2 \end{bmatrix}
\end{align}
\begin{align}
 g = \begin{bmatrix} (\Delta+\Delta_1-\Delta_2)(\Delta+\Delta_1-\Delta_2+1) \\ \Delta+2\Delta_1-\Delta_2 \end{bmatrix}
\end{align}
We expect that the coefficients $c_k$ of the $q$-expansion \eqref{fd} can be similarly computed as sums over a basis of a Verma module, 
\begin{align}
 c_k = \sum_{|L|=|L'|=k} G^{-1}_{L,L'}(\Delta) h_L(\Delta,\Delta_1,\Delta_2)h_L(\Delta,\Delta_4,\Delta_3)
\end{align}
How do we compute $g_L,h_L$? In the case of $g_L$, we have 
\begin{align}
 g_L = \frac{\left< LV_{\Delta}(0)V_{\Delta_1}(1)V_{\Delta_2}(\infty)\right>}{\left< V_{\Delta}(0)V_{\Delta_1}(1)V_{\Delta_2}(\infty)\right>}
\end{align}
This is a sphere 3-point function, as a result of sending $z\to 0$ in a sphere 4-point function. In the pillow, the corresponding limit is $\tau\to i\infty$, leading to an infinite cylinder quotiented by $\mathbb{Z}_2$. 

Proposed relation between the infinite half-pillow and the $x$-sphere:
\begin{align}
 du = \frac{dx}{\sqrt{x(1-x)}} = \frac{2dx}{\sin u} \quad , \quad 4x(1-x) = \sin^2u \quad , \quad x = \cos^2\tfrac{u}{2}
\end{align}
But this is the $z\to\infty$ limit of Eq. \eqref{dudx}, instead of $z\to 0$! 

In the infinite half-pillow, we expect
\begin{align}
 h_L = \frac{\left< LV_{\Delta}(i\infty)V_{\Delta_1}(0)V_{\Delta_2}(\pi)\right>}{\left< V_{\Delta}(i\infty)V_{\Delta_1}(0)V_{\Delta_2}(\pi)\right>}
\end{align}
It seems that for $n\geq -1$ we have 
\begin{align}
 L_{-n} V_\Delta(i\infty) = \oint_\infty e^{inu} T(u) V_\Delta(i\infty)
\end{align}
This equation cannot be valid for $n\leq -2$, or we would get too many Ward identities. Similarly, on the sphere, we cannot compute $L_2 V_\Delta(\infty) =0$ as a contour integral involving powers of $y-z_0$: if we could, how would we choose $z_0$? Naively doing contour integrals without regards for the integration measure or for the $\mathbb{Z}_2$ quotient, we get the Ward identities
\begin{align}
 0 &= -(\Delta_1-\Delta_2) + L_{-1}^{(0)} - L_{-1}^{(\pi)}
 \label{ze}
 \\
 i\Delta &= L_{-1}^{(0)} + L_{-1}^{(\pi)} 
 \\
 L_{-1}^{(\infty)} &= (\Delta_1-\Delta_2) + L_{-1}^{(0)} - L_{-1}^{(\pi)}
 \\
 L_{-n}^{(\infty)} &= n(\Delta_1+(-)^n\Delta_2) + L_{-1}^{(0)} +(-)^n L_{-1}^{(\pi)}
\end{align}
This leads to 
\begin{align}
 h_{L_{-n}} = \left\{\begin{array}{ll} (\Delta+n\Delta_1-n\Delta_2)  & (n \text{ even}) 
                      \\ (n+1)(\Delta_1-\Delta_2) & (n \text{ odd})
                     \end{array}\right.
\end{align}
This is $\Delta$-independent for $n$ odd but not for $n$ even. Should $\Delta$ really be $\Delta-\frac{c}{24}$? We also compute $h_{L_{-1}^2}$: to do this the lhs of Eq. \eqref{ze} has to be replaced with $2\Delta$, and we find 
\begin{align}
 h_{L_{-1}^2} = 4(\Delta_1-\Delta_2)(\Delta+\Delta_1-\Delta_2)
\end{align}


\section{Second-order terms}

Let us compute 
\begin{align}
 c_2 = \frac{R_{1,1}^2}{\Delta} +\frac{R_{1,2}}{\Delta-\Delta_{(1,2)}} + \frac{R_{2,1}}{\Delta-\Delta_{(2,1)}}
\end{align}
with the residues 
\begin{align}
 R_{1,1} = \tfrac12 (\Delta_1-\Delta_2)(\Delta_4-\Delta_3) 
 \end{align}
 \begin{multline}
   R_{2,1} = \frac{1}{4(\beta^4-1)}\left[(\Delta_2-\Delta_1)^2 -\tfrac12 \beta^2(\Delta_1+\Delta_2) -\tfrac{1}{16}(\beta^2-2)(3\beta^2-2)\right] 
   \\
\times \left[(\Delta_3-\Delta_4)^2 -\tfrac12 \beta^2(\Delta_3+\Delta_4) -\tfrac{1}{16}(\beta^2-2)(3\beta^2-2)\right] 
 \end{multline}
 where $c=13-6\beta^2-6\beta^{-2}$ is the central charge. 
We want to reduce $c_2$ to the same denominator. We compute 
\begin{multline}
 R_{2,1}+R_{1,2} = -\frac14 (\Delta_2-\Delta_1)^2(\Delta_3-\Delta_4)^2 +\frac{1}{16}(\Delta_1+\Delta_2)(\Delta_3+\Delta_4) 
 \\
 +\frac{1}{64} \left[(\Delta_2-\Delta_1)^2+(\Delta_3-\Delta_4)^2\right] -\frac{1}{256}(c+3)(\Delta_1+\Delta_2+\Delta_3+\Delta_4) + \frac{1}{4096}(c+1)(c+5)
\end{multline}
\begin{multline}
 4\left[\Delta_{(1,2)}R_{2,1}+\Delta_{(2,1)}R_{1,2}\right] = \frac{c-9}{8} (\Delta_2-\Delta_1)^2(\Delta_3-\Delta_4)^2 
 \\
 +\frac38 \left[(\Delta_1+\Delta_2)(\Delta_3-\Delta_4)^2+(\Delta_3+\Delta_4)(\Delta_1-\Delta_2)^2\right] 
 -\frac18(\Delta_1+\Delta_2)(\Delta_3+\Delta_4) 
 \\
 -\frac{c}{32}\left[(\Delta_2-\Delta_1)^2+(\Delta_3-\Delta_4)^2\right] +\frac{c}{128}(\Delta_1+\Delta_2+\Delta_3+\Delta_4) + \frac{c(1-c)}{2048}
\end{multline}
This leads to 
\begin{multline}
 (\Delta-\Delta_{(2,1)})(\Delta-\Delta_{(1,2)}) c_2 = R_{1,1}^2\frac{2+\frac{c}{4\Delta}}{4} + \cdots d \\
\end{multline}






\bibliographystyle{cft}
\bibliography{cft}

\end{document}

