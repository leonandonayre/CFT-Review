
\documentclass[12pt,a4paper]{article}

\RequirePackage[l2tabu, orthodox]{nag}

\usepackage[T1]{fontenc}
\usepackage[utf8]{inputenc}

\newcommand \versionno{\jobname.tex}

\usepackage{amsmath,amssymb,amsfonts,amscd,latexsym,amsthm}

\usepackage[margin=20mm]{geometry}

%\usepackage{showkeys}

\usepackage[colorlinks=true,linktoc=all,linkcolor=black,citecolor=red,urlcolor=blue]{hyperref}

\begin{document}

\typeout{}\typeout{\versionno}\typeout{} \begin{center} \fbox{\texttt{\versionno\ -- } {\small \today\ }} \end{center}

\vspace{2mm}

\begin{center}
 \textit{ \Huge Conformal blocks and the pillow geometry}
\end{center}

\tableofcontents

\section{Introduction}

The aim of this note is to provide a derivation of Zamolodchikov's formula for sphere 4-point blocks by recursion in the channel dimension. Zamolodchikov's original derivation \cite{zam87b} is based on the heavy asymptotic limit. Several steps in the derivation are hard to justify, starting with the appearance of a differential equation. Moreover, it seems overly complicated to use the heavy limit, when we are really interested in the limit where the channel dimension is large, while the other dimensions and the central charge are finite. 

An geometrical interpretation of Zamolodchikov's formula was proposed in \cite{msz15}, in terms of the pillow geometry. Here we would like to use this interpretation for deriving the formula. By deriving we mean proving although not with full mathematical rigor: in particular, we assume without proof that the conformal blocks exist and that the recursion converges. By deriving we also mean justifying as simply as possible where the formulas come from. Ideally this would involve uniquely characterizing the pillow geometry in terms of properties that imply that blocks are ``nice'' in the pillow coordinates. 

It is relatively easy to determine the poles and residues of the $s$-channel conformal block $\mathcal{F}_\Delta$ as a function of the channel dimension $\Delta$. To prove the recursion, it remains to determine the $\Delta\to\infty$ asymptotic behaviour. The basic property that we need is 
\begin{align}
 \log \mathcal{F}_\Delta = O(\Delta)
\end{align}
This is not obvious at all from the pedestrian $z$-expansion of the block, which is of the type 
\begin{align}
 \mathcal{F}_\Delta = z^{\Delta-\Delta_1-\Delta_2}\sum_{k=0}^\infty f_k z^k \quad \text{where} \quad f_k =   O(\Delta^k)
 \label{fk}
\end{align}
Taking the log must therefore lead to cancellations of terms in $O(\Delta^{j\geq 2})$. To order $z^k$ we need $k-1$ cancellations. This is already nontrivial in the case $k=2$. A direct pedestrian approach to all orders looks doomed. Maybe it would be feasible in the AGT representation of blocks, but we are looking for a more elementary approach. 

We want to prove that the block is of the type 
\begin{align}
 \mathcal{F}_\Delta = (16q)^\Delta \Omega(q)S_\Delta(q) \qquad \text{with} \qquad S_\Delta(q) =  1 + \sum_{k=1}^\infty c_k (16q)^k 
 \label{fd}
\end{align}
The property of the series $S_\Delta$ that we need can be formulated in several ways:
\begin{itemize}
 \item $\lim_{\Delta\to \infty} c_k = 0$
 \item Bulk fields are negligible as $\Delta\to\infty$, i.e. $\lim_{\Delta\to\infty} S_\Delta(q)$ is $\Delta_i$-independent (and can therefore be computed at $\Delta_i=0$). 
\end{itemize}
The function $q(z)= \frac{1}{16} \exp \lim_{\Delta\to \infty} \frac{\log \mathcal{F}_\Delta}{\Delta}$ can be deduced from the block. From this function, can we in principle deduce the pillow geometry?


\section{Pillow geometry}

The pillow is 
\begin{align}
 \left\{x,y\in \overline{\mathbb{C}}^2\middle| y^2=x(x-1)(x-z) \right\}\bmod (y\to -y)
\end{align}
The pillow metric is $dud\bar u$ where $u\equiv u+2\pi \equiv u+2\pi\tau$ is defined by 
\begin{align}
 du = \frac{1}{\theta_3(q)^2} \frac{dx}{y} \quad \text{with} \quad q=e^{i\pi\tau}
 \label{dudx}
\end{align}
Here $dx^2$ is the flat metric on the Riemann sphere $\overline{\mathbb{C}}$. The pillow is a flat torus, with the curvature concentrated at the 4 singularities $x=0,1,z,\infty$. The normalization factor $\theta_3(q)^2$ is determined by the condition that the $s$-cycle of the torus has length $2\pi$. By $s$-cycle we mean a cycle that separates $0,z$ from $1,\infty$:
\begin{align}
  \oint du = 2\pi \quad \implies \quad \oint \frac{dx}{y} =  2\pi\theta_3(q)^2 = 4K(z)
\end{align}
The correspondence between coordinates acts as:
\begin{align}
\renewcommand{\arraystretch}{1.5}
 \begin{array}{|c||c|c|c|c||c|}
 \hline 
  x,y & z & 0 & 1 & \infty & y\to -y 
  \\
  \hline 
   u &  \pi & 0 & \pi(\tau+1) & \pi\tau & u \to -u
   \\
   \hline 
 \end{array}
\end{align}
Special cases of the relations between $q,\tau,z$:
\begin{align}
 \renewcommand{\arraystretch}{1.5}
 \begin{array}{|c||c|c|c|}
 \hline 
  z & 0 & 1 & \infty 
  \\
  \hline 
  q & 0 & 1 & i 
  \\
  \hline 
  \tau & i\infty & 0 & 1 
  \\
  \hline 
 \end{array}
\end{align}
The factor $\Omega(q)$ in \eqref{fd} is deduced from the conformal transformation to $x,y\to u$. This explains in particular why it is factorized as a function of $\Delta_i$. 

\section{Sum over states}

\subsection{Principle}

In the $z$-coordinate, the behaviour \eqref{fk} of the block's coefficients follows from their calculation from Ward identities:
\begin{align}
 f_k = \sum_{|L|=|L'|=k} G^{-1}_{L,L'}(\Delta) g_L(\Delta,\Delta_1,\Delta_2)g_{L'}(\Delta,\Delta_4,\Delta_3)
\label{fke}
 \end{align}
where the sum is over creation operators $L=\prod_{i=1}^m L_{-n_i}$ of level $k=\sum n_i$, with 
\begin{align}
 \deg_\Delta g_L = m\leq k
\end{align}
For example, at level $k=1$, we have 
\begin{align}
 g_{L_{-1}} = \Delta + \Delta_1-\Delta_2 \quad , \quad G^{-1}_{L_{-1},L_{-1}} = \frac{1}{2\Delta} 
\end{align}
At level $k=2$, we have 
\begin{align}
G^{-1} = \frac{1}{16(\Delta-\Delta_{(2,1)})(\Delta-\Delta_{(1,2)})} \begin{bmatrix} 2+\frac{c}{4\Delta} & -3 \\ -3 & 4\Delta+2 \end{bmatrix}
\end{align}
\begin{align}
 g = \begin{bmatrix} (\Delta+\Delta_1-\Delta_2)(\Delta+\Delta_1-\Delta_2+1) \\ \Delta+2\Delta_1-\Delta_2 \end{bmatrix}
\end{align}
According to \cite[Eq. (7.11)]{msz15}, the series 
\begin{align}
 \widetilde{S}_\Delta(q) =  \prod_{k=1}^\infty\left(1-q^{2k}\right)^{-\frac12} S_\Delta(q)  = 1+\sum_{k=1}^\infty \tilde{c}_k (16q)^k = 1 + c_1 (16q) + \left(c_2+\tfrac{1}{512}\right) (16q)^2 + \cdots 
\end{align}
is a sum over a basis of a Verma module, i.e. 
\begin{align}
 \tilde{c}_k = \sum_{|L|=|L'|=k} G^{-1}_{L,L'}(\Delta) h_L(\Delta,\Delta_1,\Delta_2)h_L(\Delta,\Delta_4,\Delta_3)
 \label{ghh}
\end{align}
The challenge is to prove 
\begin{align}
 \lim_{\Delta\to \infty} \left(1+\sum_{k=1}^\infty \tilde{c}_k (16q)^k\right) = \prod_{k=1}^\infty\left(1-q^{2k}\right)^{-\frac12}
 \label{ltc}
\end{align}
Notice that \cite[Eq. (7.11)]{msz15} follows from the identity 
\begin{align}
 \theta_2(q)\theta_3(q)\theta_4(q) = 2q^\frac14 \prod_{k=1}^\infty (1-q^{2k})^3
\end{align}
which can be proved using the Jacobi triple product identity, as well as the identity $\prod_{k=1}^\infty (1+q^k)(1-q^{2k+1}) = 1$. (To prove the latter identity, use elementary manipulations to show that the expression is invariant under $q\to q^2$.)

\subsection{Ward identities}

How do we compute $g_L,h_L$? In the case of $g_L$, we have 
\begin{align}
 g_L = \frac{\left< LV_{\Delta}(0)V_{\Delta_1}(1)V_{\Delta_2}(\infty)\right>}{\left< V_{\Delta}(0)V_{\Delta_1}(1)V_{\Delta_2}(\infty)\right>}
\end{align}
This is a sphere 3-point function, as a result of sending $z\to 0$ in a sphere 4-point function. In the pillow, the corresponding limit is $\tau\to i\infty$, leading to an infinite cylinder quotiented by $\mathbb{Z}_2$. 

Proposed relation between the infinite half-pillow and the $x$-sphere:
\begin{align}
 du = \frac{dx}{\sqrt{x(1-x)}} = \frac{2dx}{\sin u} \quad , \quad 4x(1-x) = \sin^2u \quad , \quad x = \cos^2\tfrac{u}{2}
\end{align}
But this is the $z\to\infty$ limit of Eq. \eqref{dudx}, instead of $z\to 0$! 

In the infinite half-pillow, we expect
\begin{align}
 h_L = \frac{\left< LV_{\Delta}(i\infty)V_{\Delta_1}(0)V_{\Delta_2}(\pi)\right>}{\left< V_{\Delta}(i\infty)V_{\Delta_1}(0)V_{\Delta_2}(\pi)\right>}
\end{align}

Proposed calculation of $h_{L_{-n}}$: use a function $g_n(u)=\sum_{k\in\mathbb{Z}}g_k e^{iku} $ on the half-pillow such that $g_k=0$ unless $k=n$ or $k<0$. After conformal transformation to the sphere with 3 punctures, this corresponds to a polynomial $f_n(x)$ of degree $n+1$ that vanishes at $x=0,1$, for example 
\begin{align}
 f_1(x) = x(1-x) \quad , \quad f_2(x) = x(1-x)\left(x-\tfrac12\right) \quad ,\quad f_3(x) = x(1-x)\left(x^2-x+\tfrac{3}{16}\right)
\end{align}
A priori this leads to $h_{L_{-n}}$ being a function of $\Delta_1,\Delta_2$, independent from $\Delta$.
However, the Schwarzian term in the conformal transformation of $T$ should add some $c$-dependence if $n\geq 2$. Moreover, $\Delta$-dependence may appear when computing $h_{\prod_j L_{-n_j}}$. 

\section{Second-order terms}

\subsection{Calculation}

We introduce the notations 
\begin{align}
 \gamma = (\Delta_1-\Delta_2)^2 \ , \ \sigma = \Delta_1+\Delta_2 \ , \ \bar\gamma = (\Delta_4-\Delta_3)^2 \ , \ \bar\sigma = \Delta_4+\Delta_3 
\end{align}
and the modulus square notation $|f|^2 = f\bar f$. 
We have 
\begin{align}
 c_1 = \frac{R_{1,1}}{\Delta} = \frac{|\gamma|}{2\Delta}
\end{align}
Let us also compute 
\begin{align}
 c_2 = \frac{R_{1,1}^2}{\Delta} +\frac{R_{1,2}}{\Delta-\Delta_{(1,2)}} + \frac{R_{2,1}}{\Delta-\Delta_{(2,1)}}
\end{align}
We compute the residues 
\begin{align}
 R_{1,1} = \tfrac12 (\Delta_1-\Delta_2)(\Delta_4-\Delta_3)  = \tfrac12 |\gamma| 
 \end{align}
 \begin{align}
   R_{2,1} &= \frac{1}{4(\beta^4-1)}\left[(\Delta_2-\Delta_1)^2 -\tfrac12 \beta^2(\Delta_1+\Delta_2) -\tfrac{3}{16}\beta^4+\tfrac12\beta^2-\tfrac14\right] 
   \\
 & \hspace{2cm} \times \left[(\Delta_3-\Delta_4)^2 -\tfrac12 \beta^2(\Delta_3+\Delta_4) -\tfrac{3}{16}\beta^4+\tfrac12\beta^2-\tfrac14\right] 
 \\
  &= \frac{1}{4(\beta^4-1)}\left|\gamma -\tfrac12 \beta^2 \sigma -\tfrac{3}{16}\beta^4+\tfrac12\beta^2-\tfrac14\right|^2 
 \end{align}
 where $c=13-6\beta^2-6\beta^{-2}$ is the central charge. 
We want to reduce $c_2$ to the same denominator. We compute 
\begin{align}
 8(R_{2,1}+R_{1,2})= -2 |\gamma|^2 +\tfrac{1}{2}|\sigma|^2
 +\tfrac{1}{8} (\gamma+\bar\gamma) -\tfrac{c+3}{32}(\sigma+\bar\sigma) + \tfrac{(c+1)(c+5)}{512}
\end{align}
\begin{align}
 12(\beta^2R_{1,2}+\beta^{-2}R_{2,1}) = \tfrac{c-13}{2}|\gamma|^2 +\tfrac32(\sigma\bar\gamma+\bar\sigma\gamma)+\tfrac{1-c}{8}(\gamma+\bar\gamma) -\tfrac{3}{32}(\sigma+\bar\sigma)+\tfrac{7c+5}{512}
\end{align}
\begin{align}
 (\Delta-\Delta_{(2,1)})(\Delta-\Delta_{(1,2)}) = \Delta^2 + \tfrac{c-5}{8}\Delta + \tfrac{c}{16}
\end{align}
\begin{multline}
 16(\Delta-\Delta_{(2,1)})(\Delta-\Delta_{(1,2)}) c_2 = |\gamma|^2\left(4\Delta + \tfrac{c-5}{2} + \tfrac{c}{4\Delta}\right)
 \\
 +8(2\Delta+1)(R_{2,1}+R_{1,2}) -12(\beta^2R_{1,2}+\beta^{-2}R_{2,1})
\end{multline}
This leads to 
\begin{multline}
 16(\Delta-\Delta_{(2,1)})(\Delta-\Delta_{(1,2)}) c_2 = \left(2+\frac{c}{4\Delta}\right)|\gamma|^2 
 -\tfrac32(\gamma\bar\sigma + \bar\gamma\sigma)+\tfrac{c-1}{8}(\gamma+\bar\gamma)+\tfrac{3}{32}(\sigma+\bar\sigma)-\tfrac{7c+5}{512}
 \\
 + (2\Delta+1)\left[\tfrac12|\sigma|^2 +\tfrac18 (\gamma+\bar\gamma) -\tfrac{c+3}{32}(\sigma+\bar\sigma)+\tfrac{(c+1)(c+5)}{512}\right] \\
\end{multline}
which may be rewritten as 
\begin{multline}
 16(\Delta-\Delta_{(2,1)})(\Delta-\Delta_{(1,2)}) \left(c_2+\tfrac{1}{512}\right) 
 \\
 = \left(2+\frac{c}{4\Delta}\right)\left|\gamma+\tfrac{\Delta}{8}\right|^2 
 -3\Re\left(\gamma+\tfrac{\Delta}{8}\right)\left(\bar\sigma-\tfrac{c}{16}\right)
 +\left(\Delta+\tfrac12\right)\left|\sigma-\tfrac{c}{16}\right|^2 
\end{multline}

\subsection{Interpretation}

Up to $O(q^2)$, the explicit calculation agrees with Eq. \eqref{ghh}, provided
\begin{subequations}
 \label{hl}
\begin{align}
 h_{L_{-1}} &= \Delta_1-\Delta_2
 \\
 h_{L_{-1}^2} &= (\Delta_1-\Delta_2)^2 +\frac{\Delta}{8}
 \\
 h_{L_{-2}} &= \Delta_1+\Delta_2-\frac{c}{16}
\end{align}
\end{subequations}
Notice that $\frac{c}{16}$ is the dimension of twist fields for a conical defect of angle $\pi$, this may not be a coincidence. 

\section{Oscillator representation}

A proof of exponentiation of Virasoro blocks in the heavy limit was proposed, using the oscillator representation of the Virasoro algebra \cite{bdk19}. The oscillator representation provides a basis of a Verma module such that the Gram matrix $G$ is diagonal: using this basis, properties of blocks reduce to properties of 3-point functions of descendants $h_L$ in Eq. \eqref{ghh}. The argument of \cite{bdk19} seems to apply straightforwardly to the limit $\Delta\to\infty$ we are interested in, not just to the heavy limit. Most probably, it also applies to the more general limits of \cite{al24}. 

The oscillator representation amounts to rewriting the Virasoro algebra in terms of an abelian affine Lie algebra: 
\begin{align}
 [J_m,J_n] = -\frac12 n\delta_{m+n,0}\quad , \quad Q = \beta -\beta^{-1} \quad , \quad c = 1-6Q^2
\end{align}
\begin{align}
 L_n &= \sum_{m\in{\mathbb{Z}}} J_{n-m}J_m + Q(n+1)J_n\ , \qquad (n\neq 0)\ ,
\label{lnj}
\\
L_0 &=2\sum_{m=1}^\infty J_{-m}J_m +J_0^2+QJ_0 \ ,
\label{lzj}
\end{align}
An affine primary field $V_\alpha$ such that $J_0V_\alpha = \alpha V_\alpha$ is also a Virasoro primary, with 
\begin{align}
 \Delta = \alpha(\alpha + Q)
\end{align}
We define the conjugation 
\begin{align}
 J_n^* = -J_{-n}-Q\delta_{n,0} \implies L_n^* = L_{-n} \quad , \quad \alpha^*=-Q-\alpha
\end{align}
The diagonal matrix elements of the Gram matrix are 
\begin{align}
 G_{J_{-1}} = \left<\alpha\middle|J_{-1}^*J_{-1}\middle|\alpha\right>= -\frac12  \quad , \quad G_{J_{-2}} = -1 \quad ,\quad G_{J_{-1}^2} = \frac12
\end{align}
The relation between the oscillator basis and the Virasoro basis of a Verma module is 
\begin{align}
 J_{-1} = \frac{1}{2\alpha} L_{-1} \quad , \quad J_{-1}^2 = \frac{(2\alpha-Q)L_{-1}^2 -2\alpha L_{-2}}{2\alpha(2\alpha-\beta)(2\alpha+\beta^{-1})} \quad , \quad J_{-2} = \frac{4\alpha^2 L_{-2}-L_{-1}^2}{2\alpha(2\alpha-\beta)(2\alpha+\beta^{-1})}
 \label{jl}
\end{align}
We can compute the coefficient $f_1$ \eqref{fke} as 
\begin{align}
 f_1 = G_{J_{-1}}^{-1} g_{J_{-1}} g_{J_{-1}}^* = \frac{-2}{4\alpha(-Q-\alpha)}(\Delta+\Delta_1-\Delta_2)(\Delta+\Delta_4-\Delta_3)
\end{align}
Let us use the oscillator basis in the pillow geometry. In the large $\alpha$ limit, we compute $h_{J_{-2}},h_{J_{-1}^2}$ from $h_{L_{-2}},h_{L_{-1}^2}$ \eqref{hl} using Eq. \eqref{jl}, and we find 
\begin{align}
 \lim_{\alpha\to \infty} h_{J_{-2}} = 0 \quad , \quad \lim_{\alpha\to \infty} h_{J_{-1}^2} = \frac{1}{32}
 \label{lims}
\end{align}
where the nonzero limit comes from $h_{L_{-1}^2}$. We therefore find $\lim \tilde{c}_2 = \frac{1}{512}$ as expected. 
More generally we define 
\begin{align}
 \mathcal{J} = \prod_i J_{-k_i}^{n_i}  \quad \implies \quad |\mathcal{J}| = \sum_i n_ik_i \quad , \quad 
 H_\mathcal{J} = \lim_{\alpha\to \infty} h_\mathcal{J}
\end{align}
We compute 
\begin{align}
 G_\mathcal{J} = \prod_i G_{J_{-k_i}^{n_i}} \quad \text{with} \quad G_{J_{-k}^n} = 
 \left(-\frac{k}{2}\right)^{n} n! 
\end{align}
and we conjecture 
\begin{align}
 H_{\prod_i J_{-k_i}^{n_i}} = \prod_i H_{J_{-k_i}^{n_i}} 
\quad , \quad 
 H_{J_{-k}^{2n+1}} = 0 \quad , \quad H_{J_{-k}^{2n}} = \pm \frac{(2n)!}{n!} \left(\frac{k}{4^{2k+1}}\right)^n 
\label{hj}
 \end{align}
 in agreement with Eq. \eqref{lims}. 
Then we compute 
\begin{align}
 \sum_\mathcal{J} G_\mathcal{J}^{-1} H_\mathcal{J}^2 (16q)^{|\mathcal{J}|} = \prod_{k=1}^\infty \sum_{n=0}^\infty G^{-1}_{J_{-k}^{2n}} H_{J_{-k}^{2n}}^2 (16q)^{2kn} 
 = \prod_{k=1}^\infty \sum_{n=0}^\infty \frac{(2n)!}{(2^nn!)^2} q^{2kn} = \prod_{k=1}^\infty \left(1-q^{2k}\right)^{-\frac12}
\end{align}
Therefore, Eq. \eqref{ltc} follows from the conjecture \eqref{hj}, which was in fact crafted for that purpose. To prove the conjecture, it does not seem that the free bosonic theory can help, since in that theory our limit makes no sense, due to momentum conservation. 


\bibliographystyle{cft}
\bibliography{cft}

\end{document}

