

\documentclass[12pt, a4paper]{article}

\raggedbottom

\RequirePackage[l2tabu, orthodox]{nag}

\usepackage[top=20mm,bottom=20mm,left=25mm,right=25mm]{geometry}

%\usepackage{showkeys}

\usepackage{amsmath,amssymb,amsfonts}

\usepackage{centernot}

\usepackage{underbracket}

\usepackage[nottoc]{tocbibind}

\usepackage{pgf}
\usepackage{tikz}

\usepackage{multirow}

\usepackage{pgf}
\usepackage{tikz}

\usepackage{xcolor}

% Syntax: \colorboxed[<color model>]{<color specification>}{<math formula>}
\newcommand*{\colorboxed}{}
\def\colorboxed#1#{%
  \colorboxedAux{#1}%
}
\newcommand*{\colorboxedAux}[3]{%
  % #1: optional argument for color model
  % #2: color specification
  % #3: formula
  \begingroup
    \colorlet{cb@saved}{.}%
    \color#1{#2}%
    \boxed{%
      \color{cb@saved}%
      #3%
    }%
  \endgroup
}

\usepackage[most]{tcolorbox}

\tcbset{
    frame code={}
    center title,
    left=0pt,
    right=0pt,
    top=0pt,
    bottom=0pt,
    colback=green!15,
    colframe=white,
    % width=\dimexpr\textwidth\relax,
    enlarge left by=0mm,
    % boxsep=5pt,
    arc=0pt,outer arc=0pt,
    breakable,
    }

\usepackage{amsthm}

\usepackage{thmtools}
%\usepackage{thm-restate}   This package does not interact very smoothly with \theoremstyle and \listoftheorems

\newtheoremstyle{break}{9pt}{9pt}{\itshape}{}{\bfseries}{}{\newline}{}
\theoremstyle{break}    

\newtheorem{exo}{Exercise}[section]
\newtheorem{hyp}[exo]{Axiom}
\newtheorem{res}[exo]{Result}
\newtheorem{defn}[exo]{Definition}

\makeatletter
\def\ll@exo{%
  \protect\numberline{\csname the\thmt@envname\endcsname}%
  \ifx\@empty\thmt@shortoptarg
    \thmt@thmname
  \else
    \thmt@shortoptarg
  \fi}
\def\l@thmt@exo{} 
\makeatother

\usepackage[colorlinks=true,linktoc=all,linkcolor=black,citecolor=red,urlcolor=blue]{hyperref}

\title{\bfseries Exactly solvable two-dimensional \\ conformal field theories}

\author{Sylvain Ribault \vspace{2mm}
\\
{\normalsize CEA Saclay, Institut de Physique Th\'eorique}
 \\
 {\footnotesize \ttfamily sylvain.ribault@ipht.fr }
}

\begin{document}


\maketitle


\begin{abstract}
This course will introduce two-dimensional CFT in the bootstrap approach, and sketch the known exactly solvable CFTs with no extended chiral symmetry.
\begin{itemize}
 \item The Virasoro algebra, its representations, Ward identities, fusion rules.
 \item (Generalized) minimal models, Liouville theory, logarithmic minimal models, the $O(n)$ model and more general loop models. Taking limits in the central charge and/or in conformal dimensions. 
 \item Conformal bootstrap methods, analytic and numerical. Generic and degenerate conformal blocks. Crossing symmetry equations and their solutions. 
 \item Exactly known structure constants. Analytic properties of correlation functions. 
\end{itemize}
\end{abstract}

\vspace{5mm}


\textit{
Could make life easier by renouncing logarithms. But they appear naturally when taking limits, and are in principle needed for solving the $O(n)$ model.}





\clearpage

\tableofcontents

\hypersetup{linkcolor=blue}

\numberwithin{equation}{section}
\setcounter{section}{-1}

\section{Introduction}

CFTs that are exactly solvable with known methods = bootstrap, analytic and numerical. Only Virasoro. Do not characterize with twist gap, as symmetry cannot be inferred from spectrum: chiral symmetry fields might be absent from spectrum. Also, affine abelian symmetry can be present in spectrum but absent in interactions. (Is there a strong statement that unitary CFTs do not have these pathologies?)

Forget Coulomb gas, a dead end. Larger symmetry allows many generalizations, but tend to be always more complicated. (Liouville not solved from WZW, quite the opposite.) No modular bootstrap. 

Exact solution means analytic expression for three-point structure constants. Here: CFTs that are presumed to be solvable, not necessarily solved. 

This review: only necessary calculations. Derive the recursion for blocks? hard part is coef $R_{r,s}$ but it obeys simple shift equations. Rewrite it in terms of Upsilon functions, and somehow derive it from Liouville theory!

\subsection*{Acknowledgements}


%\end{tcolorbox}

\section{Algebraic structures}

Use low central charge notations ($\beta$ not $b$) because more examples. Rare are solved CFTs with large $c$.

\subsection{The Virasoro algebra and its highest-weight representations}

\subsection{Fields and correlation functions}

Axioms including single-valuedness, Ward identities. Degenerate fusion. Poles at degenerate channel momentums in OPE. 


\section{Sketching exactly solvable CFTs}

Diagonal fields. Non-diagonal fields. Get logarithmic fields by degenerate fusion.

Start with GMM. Then focus on rational central charge, get MM. If we take limit, we presumably get log-MM. 

Deduce Liouville with $c\leq 1$ by taking a limit. Then get Liouville for generic $c$. Other limits of diag CFT: RWT. Limit of non-diagonal MM.

Then reduce to one degenerate field. Maximal spectrum is $O(n)$. Depends on $\beta^2$, not $c$. Cannot really argue for Potts, but does not matter much. Anyway, at this stage, we know nothing about global symmetry.

Brownian loop soup: presumably has higher symmetry.

Ashkin--Teller does have higher symmetry, solved wrt Virasoro nevertheless.

\section{Bootstrap methods}

Notion of interchiral symmetry.

\subsection{Analytic bootstrap}

Non-diagonal, simplify it as much as possible. Non-diagonal solution. 

Loop weights are invariant under shifts, left undetermined. 

\subsection{Numerical bootstrap}

Sketch method. Start with Zamolodchikov recursion.

\section{Exact solutions for structure constants}

Deduce analytic properties of correlation functions, and limits of CFTs. 

Do we want a synthetic table of known solutions? 



\bibliographystyle{cft}
\bibliography{cft}

%\input{refs.tex}

\end{document}


\appendix




\end{document}

